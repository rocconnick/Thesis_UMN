%!TEX root = ../thesis_phd.tex
%%%%%%%%%%%%%%%%%%%%%%%%%%%%%%%%%%%%%%%%%%%%%%%%%%%%%%%%%%%%%%%%%%%%%%%%%%%%%%%%
% event_selection.tex: Select of showering and tracking events:
%%%%%%%%%%%%%%%%%%%%%%%%%%%%%%%%%%%%%%%%%%%%%%%%%%%%%%%%%%%%%%%%%%%%%%%%%%%%%%%%
\chapter{Event Selection}
\label{event_selection_chapter}
%%%%%%%%%%%%%%%%%%%%%%%%%%%%%%%%%%%%%%%%%%%%%%%%%%%%%%%%%%%%%%%%%%%%%%%%%%%%%%%%

Physical parameters can only be measured using a clean sample of events
which exhibit the process under study.
The definition of clean can of course vary wildly depending on the
experiment and parameters in question.
For this analysis, the goal is to accurately determine the reconstructed energy
spectrum for \numu CC events.
Thus, the goal of event selection is twofold: the selected sample should be
relatively background free and consist of events for which the energy of the
interacting neutrino can be determined.

Estimating the energy of events requires that they be fully contained
within the detector.
If all particles borne in the interaction deposit their energy within
the detector, the a estimate of the neutrino energy can be obtained using
the method described in section \ref{energy_section}.
If one or more of the particles escape the detector, it can be difficult
to estimate how much energy is missing.
Events with poor energy resolution can provide some sensitivity to oscillation
parameters, in particular \thetatth, which governs the amplitude of oscillation.
This analysis, however, makes no attempt at including escaping events.

Two primary backgrounds which must be suppressed for this analysis.
The first is the vastly dominant cosmic ray background; the FD not underground
and is exposed to cosmic rays at a rate exceeding 100 kHz.
The second is source of background is events from the \numi beam which are
not \numu CC interactions.
NC interactions are the trickiest beam background to reject due to their
abundance, especially those with a charged pion in the final state which
can produce a relatively long track.
Both appeared and survived \nue in the beam are relatively rare and tend
not to fake \numu CC signal.


\section{Cosmic Ray Preselection}
\label{cosmicveto_section}

%%%%%%%%%%%%%%%%%%%%%%%%%%%%%%%%%%%%%%%%%%%%%%%%%%%%%%%%%%%%%%%%%%%%%%%%%%%%%%%%
