%!TEX root = ../thesis_phd.tex
%%%%%%%%%%%%%%%%%%%%%%%%%%%%%%%%%%%%%%%%%%%%%%%%%%%%%%%%%%%%%%%%%%%%%%%%%%%%%%%%
% app_glossary.tex: Glossary Appendix:
%%%%%%%%%%%%%%%%%%%%%%%%%%%%%%%%%%%%%%%%%%%%%%%%%%%%%%%%%%%%%%%%%%%%%%%%%%%%%%%%
\chapter{Glossary and Acronyms}
\label{app_glossary}
%%%%%%%%%%%%%%%%%%%%%%%%%%%%%%%%%%%%%%%%%%%%%%%%%%%%%%%%%%%%%%%%%%%%%%%%%%%%%%%%
Care has been taken in this thesis to minimize the use of jargon and
acronyms, but this cannot always be achieved.  This appendix defines
jargon terms in a glossary, and contains a table of acronyms and their
meaning.
%%%%%%%%%%%%%%%%%%%%%%%%%%%%%%%%%%%%%%%%%%%%%%%%%%%%%%%%%%%%%%%%%%%%%%%%%%%%%%%%

%%%%%%%%%%%%%%%%%%%%%%%%%%%%%%%%%%%%%%%%%%%%%%%%%%%%%%%%%%%%%%%%%%%%%%%%%%%%%%%%
% Glossary {{{
%%%%%%%%%%%%%%%%%%%%%%%%%%%%%%%%%%%%%%%%%%%%%%%%%%%%%%%%%%%%%%%%%%%%%%%%%%%%%%%%
\section{Glossary}
\label{jargonapp}
%%%%%%%%%%%%%%%%%%%%%%%%%%%%%%%%%%%%%%%%%%%%%%%%%%%%%%%%%%%%%%%%%%%%%%%%%%%%%%%%
\begin{itemize}

\item \textbf{Analog to Digital Converter} (\textbf{ADC}) -- A device which measures an analog response and produces a discretized value as output.

\item \textbf{Baryon} -- A particle comprised of three valence
quarks.  Protons and neutrons are examples of baryons.

\item \textbf{Cell} -- The fundamental unit of \nova detectors.  Cells
are filled with liquid scintillator and use a wavelength shifting fiber
to transmit light to an avalanche photodiode (APD).

\item \textbf{Cell Hit} -- A recorded pulse of light recorded in a \nova cell.

\item \textbf{Charge-Parity (CP)} -- The combined operation of charge
reversal (particles to antiparticles) and spatial inversion (reversal of
coordinates).

\item \textbf{Hadron} -- A composite particle made of at least two quarks.  \textit{Mesons} such as pions and kaons are composed of one quark and one antiquark.  \textit{Baryons} are formed by some combination of three quarks or antiquarks.  Recently, more exotic states with four or five quarks have been observed in medium energy electron accelerators \cite{dias2013z_,barth2003evidence}.

\item \textbf{Meson} -- A particle comprised of two valence
quarks.  Pions and kaons are examples of baryons.

\item \textbf{Monte Carlo (MC) truth} -- Detailed information regarding
particle interactions and detector response in MC simulation.

\item \textbf{Muon Catcher} -- The block furthest downstream in the \nova
ND which is used to stop (range-out) muons before they exit the detector.
Active \nova planes are interleaved with 10 cm steel planes in order to provide
additional stopping power.

\item \textbf{Quantum Efficiency} -- The fraction of photons which produce photo-electrons in a piece of light detection hardware such as a photo-diode or photomultiplier tube.

\item \textbf{Slice} -- A cluster of hits which is produced by the \nova
implementation \cite{baird2015thesis} of the DBSCAN \cite{ester1996density}
algorithm (Section~\ref{slicer_section}).

\item \textbf{Track} -- A reconstructed particle trajectory.  Algorithms
which locate tracks are described in Sections \ref{cosmictrack_section} and
\ref{kalmantrack_section}.

\item \textbf{Range-out} -- A particle is said to range-out when it runs out of
energy after depositing it all through ionization.

\end{itemize}
%%%%%%%%%%%%%%%%%%%%%%%%%%%%%%%%%%%%%%%%%%%%%%%%%%%%%%%%%%%%%%%%%%%%%%%%%%%%%}}}

%%%%%%%%%%%%%%%%%%%%%%%%%%%%%%%%%%%%%%%%%%%%%%%%%%%%%%%%%%%%%%%%%%%%%%%%%%%%%%%%
% Acronyms {{{
%%%%%%%%%%%%%%%%%%%%%%%%%%%%%%%%%%%%%%%%%%%%%%%%%%%%%%%%%%%%%%%%%%%%%%%%%%%%%%%%
\section{Acronyms}
\label{acronymsec}
%%%%%%%%%%%%%%%%%%%%%%%%%%%%%%%%%%%%%%%%%%%%%%%%%%%%%%%%%%%%%%%%%%%%%%%%%%%%%%%%

%\setlength\LTleft{0pt}
%\setlength\LTright{0pt}

\begin{longtable}{|p{0.25\textwidth}|p{0.75\textwidth}|}
\caption{Acronyms} \label{Acronyms} \\

\hline
Acronym & Meaning \\
\hline \hline
\endfirsthead

\multicolumn{2}{l}%
{{\bfseries \tablename\ \thetable{} -- continued from previous page}} \\
\hline
Acronym & Meaning \\
\hline \hline
\endhead

\hline \hline \multicolumn{2}{|r|}{{Continued on next page}} \\ \hline
\endfoot

\hline \hline
\endlastfoot
ADC & Analog-to-digital converter \\
APD & Avalanche photodiode \\
BDT & Boosted decision tree  \\
CC & Charged current \\
CNN & Convolutional neural network \\
CP & Charge-Parity \\
DAQ & Data acquisition \\
DCM & Data concentrator module \\
DCS & Dual correlated sampling \\
DIS & Deep-inelastic scattering \\
FD & Far Detector \\
FLS & Fiber and liquid scintillator \\
GPU & Graphical processing unit  \\
kNN & k-Nearest neighbor  \\
LRN & Local response normalization \\
MC & Monte Carlo \\
NC & Neutral current \\
ND & Near Detector \\
PVC & Polyvinyl chloride \\
POT & Protons on-target \\
QE & Quasi-elastic (scattering) \\
RES & Resonant-elastic scattering \\
SGD & Stochastic gradient descent \\
TDC & Time-to-digital converter  \\
WLS & Wavelength shifting \\

\end{longtable}
%%%%%%%%%%%%%%%%%%%%%%%%%%%%%%%%%%%%%%%%%%%%%%%%%%%%%%%%%%%%%%%%%%%%%%%%%%%%%}}}
