%!TEX root = ../thesis_phd.tex
%%%%%%%%%%%%%%%%%%%%%%%%%%%%%%%%%%%%%%%%%%%%%%%%%%%%%%%%%%%%%%%%%%%%%%%%%%%%%%%%
%%%%%%%%%%%%%%%%%%%%%%%%%%%%%%%%%%%%%%%%%%%%%%%%%%%%%%%%%%%%%%%%%%%%%%%%%%%%%%%%
% systematics.tex: systematic errors
%%%%%%%%%%%%%%%%%%%%%%%%%%%%%%%%%%%%%%%%%%%%%%%%%%%%%%%%%%%%%%%%%%%%%%%%%%%%%%%%
\chapter{Systematic Errors}
\label{systs_chapter}
%%%%%%%%%%%%%%%%%%%%%%%%%%%%%%%%%%%%%%%%%%%%%%%%%%%%%%%%%%%%%%%%%%%%%%%%%%%%%%%%

The extrapolated prediction depends on our knowledge of particle propagation
and detector propagation, mainly through MC simulation.
Such dependencies introduce systematic errors when that knowledge is incomplete.
The design of \nova and its reconstruction chain is aimed at mitigating
systematic errors.
For instance, a two detector design helps
cancel flux and cross section uncertainties; estimating muon energy
from range helps eliminate some dependence on precise calibration.
Despite these efforts, the degree to which systematic errors are mitigated
must be tested and and their effect must be incorporated into the
analysis framework.


\section{Treatment of Systematic Errors}

Systematic errors are incorporated by allowing
event records to be altered before they are handled by the
selection and analysis framework.
This process allows the analysis framework to see the analysis as each
event were different in some systematic fashion.
Each event can also be assigned an alternate weight based
on entries in the event record, such as MC truth information or
the output of the reconstruction algorithms.
For each systematic error, alternate versions of the FD prediction
are generated corresponding to $\pm 1\sigma, \pm2\sigma,$ and
$\pm3\sigma$ shifts.

The event records which are allowed to be shifted are populated by
the output of the reconstruction algorithms,
not the raw hit-by-hit detector readout itself.
Meanwhile, many of the systematic effects must be assessed by altering the
hit-by-hit detector readout.
In these cases, alternative MC samples are generated which
correspond to some systematic change in the behavior of the
simulation and/or reconstruction.
The effect of that systematic change is then
and parametrized in terms of some entry in the event record which
captures the behavior.

For each systematic error, the shape of the
$\pm 1\sigma, \pm2\sigma,$ and
$\pm3\sigma$ shifts is interpolated using a cubic spline
\cite{atkinson1978introduction}.
The interpolation gives the fitter a smooth profile for marginalization.
In the Feldman-Cousins procedure \cite{feldman1998unified} (described
in Section~\ref{feldman_cousins_section}) that shape
is used for sampling the errors in the pseudo-experiments.

\section{Flux Uncertainties}
\label{flux_syst_section}

Come back to this one.


\section{Cross Section Uncertainties}

\genie includes a built in utility for reweighting events based on
uncertainties the free parameters of its cross section models
\cite{genie}.
For each free parameter, event weights are determined by recalculating the
cross section with that parameter altered;
the weight is taken to be the ratio of the altered cross section
to the altered one.
At analysis time, selected events in the ND and FD MC predictions can be
reweighted on an event-by-event basis.

There are 67 parameters which can be adjusted independently which are listed
in Table~\ref{genie_reweight_knobs}.
In particular, certain parameters are matched with shape-only
and/or normalization-only counterparts.
The shape-only and normalization-only parameters, listed in Table
 are excluded 


 \begin{longtable}{|p{0.4\textwidth}|p{0.6\textwidth}|}
\hline
Parameter & Description \\
\hline \hline
\endfirsthead

\multicolumn{2}{l}%
{{\bfseries \tablename\ \thetable{} -- continued from previous page}} \\
\hline
Parameter & Description \\
\hline \hline
\endhead

\hline \hline \multicolumn{2}{|r|}{{Continued on next page}} \\ \hline
\endfoot

\hline
\endlastfoot


   \hline
 ReweightAGKY\_pT1pi & AGKY transverse momentum in single pion states \\ \hline
 ReweightAGKY\_xF1pi & AGKY Feynman x for single pion states \\ \hline
 ReweightAhtBY & Higher-twist parameter ($A_{HT}$) in Bodek-Yang model \\ \hline
 ReweightAhtBYshape & Higher-twist parameter ($A_{HT}$) in Bodek-Yang model, shape only \\ \hline
 ReweightBR1eta & Branching ratio for single-$\eta$ resonance decays \\ \hline
 ReweightBR1gamma & Branching ratio for radiative resonance decays \\ \hline
 ReweightBhtBY & Higher-twist parameter ($B_{HT}$) in Bodek-Yang model \\ \hline
 ReweightBhtBYshape & Highter-twist parameter ($B_{HT}$) in Bodek-Yang model, shape only \\ \hline
 ReweightCCQEMomDistroFGtoSF & CCQE Nucleon Momentum Distribution \\ \hline
 ReweightCCQEPauliSupViaKF & CCQE Pauli suppression (changes Fermi level $k_F$) \\ \hline
 ReweightCV1uBY & $C_{V1u}$  valence GRV98 PDF correction in Bodek-Yang Model \\ \hline
 ReweightCV1uBYshape & $C_{V1u}$  valence GRV98 PDF correction in Bodek-Yang Model, shape only \\ \hline
 ReweightCV2uBY & $C_{V2u}$  valence GRV98 PDF correction in Bodek-Yang Model \\ \hline
 ReweightCV2uBYshape & $C_{V2u}$  valence GRV98 PDF correction in Bodek-Yang Model, shape only \\ \hline
 ReweightDISNuclMod & DIS nuclear modification (shadowing, anti-shadowing, EMC) \\ \hline
 ReweightEtaNCEL & Strange axial form factor for NC elastic  \\ \hline
 ReweightFormZone & Formation zone \\ \hline
 ReweightFrAbs\_N & Nucleon absorption probability \\ \hline
 ReweightFrAbs\_pi &Pion absorption probability \\ \hline
 ReweightFrCEx\_N & Nucleon charge exchange probability \\ \hline
 ReweightFrCEx\_pi & Pion charge exchange probability \\ \hline
 ReweightFrElas\_N & Nucleon elastic reaction probability \\ \hline
 ReweightFrElas\_pi & Pion elastic reaction probabbility \\ \hline
 ReweightFrInel\_N & Nucleon inelastic reaction probability \\ \hline
 ReweightFrInel\_pi &Pion inelastic reaction probability \\ \hline
 ReweightFrPiProd\_N &Nucleon-pion production probability  \\ \hline
 ReweightFrPiProd\_pi &Pion-pion production probability \\ \hline
 ReweightMFP\_N & Nucleon mean free path \\ \hline
 ReweightMFP\_pi & Pion mean free path \\ \hline
 ReweightMaCCQE & Axial mass for CC quasi-elastic \\ \hline
 ReweightMaCCQEshape & Axial mass for CC quasi-elastic, shape only \\ \hline
 ReweightMaCCRES & Axial mass for CC resonance production \\ \hline
 ReweightMaCCRESshape & Axial mass for CC resonance production, shape only \\ \hline
 ReweightMaCOHpi & Axial mass for CC coherent pion production \\ \hline
 ReweightMaNCEL & Axial mass for NC elastic \\ \hline
 ReweightMaNCRES & Axial mass for NC resonance production \\ \hline
 ReweightMaNCRESshape & Axial mass for NC resonance production, shape only \\ \hline
 ReweightMvCCRES & Vector mass for CC quasi-elastic \\ \hline
 ReweightMvCCRESshape & Vector mass for CC quasi-elastic, shape only \\ \hline
 ReweightMvNCRES & Vector mass for CC resonance production \\ \hline
 ReweightMvNCRESshape & Vector mass for CC resonance production, shape only \\ \hline
 ReweightNC & NC scale... not implemented in same fashion as others, considered broken \\ \hline
 ReweightNormCCQE & CCQE normalization \\ \hline
 ReweightNormCCQEenu & CCQE normalization \\ \hline
 ReweightNormCCRES & CC resonance production, normalization \\ \hline
 ReweightNormDISCC & CC DIS production, normalization  \\ \hline
 ReweightNormNCRES & NC resonance production, normalization  \\ \hline
 ReweightR0COHpi & Nuclear size parameter controlling pion absorption  in Rein-Sehgal model  \\ \hline
 ReweightRnubarnuCC & $\nu\bar{\nu} $ CC ratio\\ \hline
 ReweightRvbarnCC1pi & Non resonance background in $ \bar{\nu} n CC 1\pi$ interactions \\ \hline
 ReweightRvbarnCC2pi & Non resonance background in $ \bar{\nu} n CC 2\pi$ interactions \\ \hline
 ReweightRvbarnNC1pi & Non resonance background in $ \bar{\nu} n NC 1\pi$ interactions \\ \hline
 ReweightRvbarnNC2pi & Non resonance background in $ \bar{\nu} n NC 2\pi$ interactions \\ \hline
 ReweightRvbarpCC1pi & Non resonance background in $ \bar{\nu} p CC 1\pi$ interactions \\ \hline
 ReweightRvbarpCC2pi & Non resonance background in $ \bar{\nu} p CC2\pi$ interactions \\ \hline
 ReweightRvbarpNC1pi & Non resonance background in $ \bar{\nu} p NC1\pi$ interactions \\ \hline
 ReweightRvbarpNC2pi & Non resonance background in $ \bar{\nu} p NC2\pi$ interactions \\ \hline
 ReweightRvnCC1pi & Non resonance background in $\nu n CC1\pi$ interactions \\ \hline
 ReweightRvnCC2pi & Non resonance background in $\nu n CC2\pi$ interactions \\ \hline
 ReweightRvnNC1pi & Non resonance background in $\nu n NC1\pi$ interactions \\ \hline
 ReweightRvnNC2pi & Non resonance background in $\nu n NC2\pi$ interactions  \\ \hline
 ReweightRvpCC1pi & Non resonance background in $\nu p CC1\pi$ interactions \\ \hline
 ReweightRvpCC2pi & Non resonance background in $\nu p CC2\pi$ interactions  \\ \hline
 ReweightRvpNC1pi & Non resonance background in $\nu p NC1\pi$ interactions \\ \hline
 ReweightRvpNC2pi & Non resonance background in $\nu p NC2\pi$ interactions \\ \hline
 ReweightTheta\_Delta2Npi & Pion angular distribution in Rein-Sehgal \\ \hline
 ReweightVecCCQEshape & Choice of CCQE vector form factors (BBA05/Dipole), shape only  \\ \hline
\caption{ All reweightable free parameters in \genie }{
\genie includes a built-in utility for reweighting events to account
for cross section uncertainties.
Free parameters are adjusted in order to calculate an alternative
cross section; weights are formed by the ratio of the alternative cross
section to the original cross section.
This table shows the complete list of parameters which can be reweighted.
}
\label{genie_reweight_knobs}
\end{longtable}





\begin{table}
\begin{center}
\begin{tabular}{|l|l|}
\hline
\textbf{Redundant Parameter} & \textbf{Counterpart} \\ \hline
MaCCQEshape &  MACCQE \\ \hline
MaCCRESshape &  MaCCRES \\ \hline
MvCCRESshape &  MvCCRES \\ \hline
MaNCRESshape &  MaNCRes  \\ \hline
MvNCRESshape &  MvNCRes \\ \hline
AhtBYshape &  AhtBY \\ \hline
BhtBYshape &  BhtBY \\ \hline
CV1uBYshape &  Cv1uBY\\ \hline
CV2uBYshape &  CV2uBY \\ \hline
NormCCQE &  MaCCQE\\ \hline
NormCCQEenu &  MaCCQE\\ \hline
NormCCRES &  MaCCRES \\ \hline
NormNCRES &  MaNCRes\\ \hline
\end{tabular}
\end{center}
\caption{Redundant knobs in \genie Reweight}{
Certain parameters in \genie are paired with shape-only or normalization-only
counterparts.
These counterparts are excluded in this analysis since they are redundant.
This table shows the list of parameters which are excluded and the parameter
with which they are redundant.
}
\label{redundant_genie_knobs}
\end{table}





