%!TEX root = ../thesis_phd.tex
%%%%%%%%%%%%%%%%%%%%%%%%%%%%%%%%%%%%%%%%%%%%%%%%%%%%%%%%%%%%%%%%%%%%%%%%%%%%%%%%
% oscillation_analysis.tex: Analysis of Neutrino Oscillations:
%%%%%%%%%%%%%%%%%%%%%%%%%%%%%%%%%%%%%%%%%%%%%%%%%%%%%%%%%%%%%%%%%%%%%%%%%%%%%%%%
\chapter{Analysis}
\label{analysis_chapter}
%%%%%%%%%%%%%%%%%%%%%%%%%%%%%%%%%%%%%%%%%%%%%%%%%%%%%%%%%%%%%%%%%%%%%%%%%%%%%%%%

A measurement of parameters \deltamtht and \thetatth is obtained from
a likelihood fit to the FD data, namely the estimated energy spectrum.
Performing such a fit requires a prediction of the FD spectrum which can be
varied in terms of the oscillation parameters.
A naive approach would be to generate a FD prediction straight from
FD Monte Carlo (MC) simulation.
That approach, however, would be quite sensitive to systematic errors,
in particular due to uncertiantes in the \numi flux and neutrino
cross sections.
\nova's two-detector design is aimed precisely at mitigating the effects of
such uncertainties; the ND can be used to determine the expected
spectrum at the FD in a way which mostly cancels the large uncertanties the
the MC prediction.
The procedure of generating the FD prediction in terms of the observed
ND spectrum, referred to as \textit{extrapolation}, is described in Section
\ref{extrap_section}.
The fitting procedure is described in Section \ref{fitting_section}.


\section{Extrapolation}
\label{extrap_section}

\nova has been designed with two detectors

Effects cause FD and ND to differ

  angle subtended

  Shape of detector



\section{Fitting}
\label{fitting_section}

binned maximum likelihood 

three flavor model

number of bins

systematic errors and constraints on unmeasured oscillation parameters
are included in the fit with a penalty term

constraints on osc. params taken from PDG

The allowed region is calculated using the FC technique.

Also describe binned maximum likelihood.



%%%%%%%%%%%%%%%%%%%%%%%%%%%%%%%%%%%%%%%%%%%%%%%%%%%%%%%%%%%%%%%%%%%%%%%%%%%%%}}}
