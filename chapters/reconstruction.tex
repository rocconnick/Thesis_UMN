%!TEX root = ../thesis_phd.tex
%%%%%%%%%%%%%%%%%%%%%%%%%%%%%%%%%%%%%%%%%%%%%%%%%%%%%%%%%%%%%%%%%%%%%%%%%%%%%%%%
% reconstruction.tex:
%%%%%%%%%%%%%%%%%%%%%%%%%%%%%%%%%%%%%%%%%%%%%%%%%%%%%%%%%%%%%%%%%%%%%%%%%%%%%%%%
\chapter{Event Reconstruction}
\label{reconstruction_chapter}
%%%%%%%%%%%%%%%%%%%%%%%%%%%%%%%%%%%%%%%%%%%%%%%%%%%%%%%%%%%%%%%%%%%%%%%%%%%%%%%%

Raw data must be processed in order to extract useful physics information for analysis.
This processing step is known as reconstruction.
For this analysis, the signal events are charged-current (CC) \numu  interactions from the \numi beam.  Cosmic rays are an abundant background which must be rejected.
This is the chapter on reconstruction


\section{Slicing}

\section{Calibration}


\section{Image Formation}

\section{Architecture and Training}

We use siamese googlenet, two googlenets side-by-side

We train first for all event types



%%%%%%%%%%%%%%%%%%%%%%%%%%%%%%%%%%%%%%%%%%%%%%%%%%%%%%%%%%%%%%%%%%%%%%%%%%%%%%%%
