%!TEX root = ../thesis_phd.tex
%%%%%%%%%%%%%%%%%%%%%%%%%%%%%%%%%%%%%%%%%%%%%%%%%%%%%%%%%%%%%%%%%%%%%%%%%%%%%%%
% intro.tex: Introduction to the thesis
%%%%%%%%%%%%%%%%%%%%%%%%%%%%%%%%%%%%%%%%%%%%%%%%%%%%%%%%%%%%%%%%%%%%%%%%%%%%%%%%
\chapter{Introduction}
\label{intro_chapter}
%%%%%%%%%%%%%%%%%%%%%%%%%%%%%%%%%%%%%%%%%%%%%%%%%%%%%%%%%%%%%%%%%%%%%%%%%%%%%%%%

%\section{Overview}

This work aims to further the understanding of a type of particle called
the neutrino, which is arguably the most mysterious of all known particles.
Neutrinos are neutral particles of relatively small mass which are only
known to interact through the weak and gravitational forces.
Since they only couple to the subdominant forces, neutrino interactions are
astonishingly rare; a neutrino could travel through a light year of lead and
have less than a 50\% chance of interacting \cite{petit2013heart}.
From the perspective of particle
physicists, the low probability of observing neutrino interactions has made
their characterization an extremely challenging task.

Neutrinos undergo a curious phenomenon called oscillation, or the ability for
members of the neutrino family to transform into other members.  The \nova
(\numi Off-axis Neutrino Appearance)
experiment has built two detectors separated by a distance of nearly 1,000 km
in order to observe neutrino oscillation.  This work aims to better measure
the frequency and amplitude of one neutrino oscillation mode, particularly by
improving the
method by which neutrinos are identified within \nova detectors.


The data output from \nova detectors can be interpreted roughly as top and side
view images of neutrino interactions.  In that sense, identification of
neutrino interactions in \nova detectors can be seen as an image classification
task.  The image classification community has made great strides through
convolutional neural network classifiers.  Implementing that technique for
interaction classification in \nova will be the central thrust of this
dissertation.



\section{The Standard Model}

High energy particle physics seeks to address the age-old topic
of describing the
constituents of matter throughout the universe.  The Greek atomist theories are
among the earliest attempts at answering this question.
Democritus is well
known for putting forth the theory that all matter was made up of indivisible,
or fundamental, particles called atoms;  Plato furthered this idea by proposing
that the all matter was comprised of four basic elements: earth, air, fire and
water \cite{berryman2008atomism}.  In the 19th century,
Dimitri Mendeleev made huge strides in
understanding the structure of matter through the organization of the periodic
table \cite{halzen1984quarks}.
The elemental atoms represented in the periodic table, however, are not
fundamental.  The Standard Model of particle physics
represents our latest attempt to describe matter in terms of fundamental
particles.  Of course, there is no guarantee that the particles of the Standard
Model are truly fundamental, but no experiment has been performed to suggest
otherwise.  A graphic which displays all of the particles in the Standard Model
can be seen in Figure \ref{sm}.



\begin{figure}
  \begin{center}
    \includegraphics[width=\textwidth]{figures/figures/sm.png}
  \end{center}
  \caption{Particles in the
  Standard Model}{
  There are six quarks: up, down, charm, strange, top, and
  bottom.
  The three charged leptons -- $e$, $\mu$, $\tau$ -- each have a neutral partner
  in the form of a neutrino -- \nue, \numu, \nutau, respectively.
  Interactions in the Standard Model are mediated by force-carrying bosons.
  The photon ($\gamma$) mediates the electromagnetic interaction, the gluon
  ($g$) mediates the strong interaction, and the \wb and \zb bosons
  mediate the weak interaction.
  The Higgs boson is responsible for imbuing particles with mass.
  Image reproduced courtesy of \cite{smWikiCitation}.
  }
  \label{sm}
\end{figure}

In the Standard Model, a particle is an entity with intrinsic charge, mass, and
spin\footnote{Spin is the property of a particle which represents its intrinsic
angular momentum.  The particles are not actually spinning in a physical
sense, although they behave as if they were \cite{halzen1984quarks}.}.
Particles in the Standard Model form two distinct categories: fermions and
bosons.
In units of $\hbar$, fermions have a half-integer spin, e.g.
$\frac{1}{2}, \frac{3}{2}$, while bosons have integer spin.  Matter is primarily made up of the fermions in the Standard Model, all of which have a
spin of~$\frac{1}{2}$.
The bosons in the Standard Model are force carriers, or in other words, they
mediate the interactions of the electromagnetic, strong and weak forces.  All
of the bosons in the Standard Model have a spin of $1$ or $0$.  Each
charged particle in the Standard Model is complimented by an antiparticle which
has the same mass and spin, but opposite charge \cite{halzen1984quarks}.

Fermions are further grouped into two categories: quarks and leptons.  Atomic
nuclei are built up from quarks; a proton is comprised of quarks and one down
quarks, a neutron is comprised of two down quarks and one up quark.
Combinations of three quarks are known as baryons, while those of two quarks
are called mesons.  Recent experiments have revealed particles which are likely
to contain four or five quarks \cite{dias2013z_,barth2003evidence}.  Distinct
baryons and mesons are not merely formed by different combinations of quarks,
but also from excited states of those combinations.  Quarks in baryons and
mesons are held together by the strong force, although they also interact
through the weak and electromagnetic force.  Antiparticle compliments of quarks
are known as antiquarks   \cite{halzen1984quarks}.

Leptons, on the other hand, do not interact through the strong force.  The most
familiar example of a lepton is the electron, which make up the exterior region
of atoms in matter.  The electron (e) has two heavier counterparts, the muon
($\mu$) and tau ($\tau$).  These three particles, along with their
antiparticles, represent the charged leptons.  Each of the charged leptons can
couple to a corresponding neutrino in weak interactions; thus the neutrinos are
named: electron neutrino (\nue), muon neutrino (\numu), and tau neutrino
($\nu_\tau$).

\section{History of Neutrino Physics}

\subsection{The Neutrino Hypothesis}

Mankind's first glimpse at the neutrino came in the early part of the 20th
century along with studies of nuclear beta decay.  Discovered in 1896 by Henri
Bequerel, beta decay was characterized by the mysterious ``beta rays" that were
emitted by various types of radioactive material \cite{becquerelBeta}.  By
1900, however, Becquerel had categorized beta rays as electrons, suggesting
that beta decay involved a heavy atom emitting a relatively light, but
fast-moving electron\footnote{
We now know beta decay to involve a neutron within a nucleus decaying into a
proton, electron and electron antineutrino, as follows:
\begin{equation*}      n \rightarrow p + e^- + \bar{\nu}_e  . \end{equation*}
Even though electrons were discovered in 1897 by J.J. Thompson in his
examination of cathode rays \cite{thompson}, atomic structure was still
shrouded in uncertainty.  Geiger, Mardsen and Rutherford's gold foil experiment
found evidence of the nucleus 1909, but neutrons were not hypothesized until
1920 by Rutherford \cite{rutherfordNeutron} and remained undiscovered until
work done in 1932 by James Chadwick. \cite{chadwickNeutron}
 } \cite{becquerelElec}.
For a two-body decay, energy conservation can predict the energy of
each particle.
In the case of beta decay, emission of a relatively lightweight electron from
a heavy atom should leave the atom roughly stationary,
thus approximately determining the energy of the electron.
James Chadwick measured the energy spectrum of beta decay electrons in 1914 and
found that the spectrum was continuous rather than discrete and single-valued
\cite {chadwickBeta}.  In other words, the measured range of electron energies
was much larger than the uncertainty in atomic recoil energy.
This result was clearly inconsistent with the two-particle process that had
previously been hypothesized.

As a solution to this problem, in 1930 Wolfgang Pauli proposed that this decay
could in fact be a three-body decay involving a third particle, which he then
called a neutron, electrically neutral and also considerably lighter than a
nucleus \cite{pauliNeuProp}.  Two years later the actual neutron we refer to
today was discovered by Chadwick \cite{chadwickNeutron} and in 1934 Enrico
Fermi proposed that Pauli's particle be called a ``neutrino."
\cite{fermiNeuName}

\subsection{Neutrino Discovery -- Methods and Observations}
\label{discovery}


Cowan and Reines made the first observation of neutrinos in 1956
\cite{cowanNature}.
The neutrinos they observed were produced through beta decays in a nuclear
reactor.  Since the field of particle physics had advanced considerably by this
point, Cowan and Reines had enough information to expect the signal:
\begin{equation} \label{beta} \bar{\nu}_e + p \rightarrow n + e^+.  \end{equation}
Detectors used in their experiment employed scintillation counters to register
energy deposition from candidate events.  Neutrino events were selected by
coincidence between a prompt electron-positron annihilation signal and a larger delayed signal produced when the neutron is captured by an atomic
nucleus, as shown in Figure \ref{oscilloscope}.
The delay of the second signal spans the time for the fast-moving neutrons to
thermalize, that is, to lose energy through nuclear scattering until reaching
the energy scale required for capture.

\begin{figure}
  \begin{center}
    \includegraphics[width=0.6\textwidth]{figures/figures/cowanOscilloscope.png}
  \end{center}
  \caption{Oscilloscope trace from Cowan and Reines' experiment}{ The small
  bump is the electron/positron annihilation signal, the large spike is the
  neutron capture signal.  Figure reproduced from \cite{cowanNature}.}
  \label{oscilloscope}
\end{figure}
Cowan and Reines measured a neutrino rate of 2.88 $\pm$ 0.22 counts/hour,
representing a signal rate 3 times larger than the measured background
 \cite{cowan}.
The neutrinos observed in this experiment were later classified as
electron ``flavor''  neutrinos.

Neutrino physics grew in scope  with the discovery of a second class of
neutrinos in 1962 \cite{numuDiscovery}.
Lee and Yang made a prediction that there existed neutrinos
associated with muons in addition to those associated with electrons
\cite{lee1957parity}.
They also made detailed calculations predicting cross sections for interaction
\eqref{beta} as well as:
\begin{equation} \label{betaMinus} \nu_e + n^0 \rightarrow p^+ + e^- \end{equation}
\begin{equation} \label{betaMuMinus}\nu_\mu + n^0 \rightarrow  p^+ + \mu^- \end{equation}
\begin{equation} \label{betaMuPlus}\bar{\nu}_\mu + p^+ \rightarrow n^0 + \mu^+ . \end{equation}


An experiment to observe the muon neutrino was devised based on the
Alternating Gradient Synchrotron (AGS) at Brookhaven National Lab.  AGS accelerated protons for collision upon a fixed target, resulting in production
of charged pions.
The pions would then decay to yield muons and muon neutrinos as follows:
\begin{equation} \label{pions} \begin{split}
\pi^+ \rightarrow \mu^+ + \nu_\mu {\ } \\
\pi^- \rightarrow \mu^- + \bar{\nu}_\mu.
\end{split} \end{equation}
An arrangement of lead and steel was placed in the beamline to absorb the
muons.  Beyond the absorbers, spark chambers were deployed that would create
sparks along the path traversed by charged particles.
Neutrinos, unaffected by the muon absorbers, would very occasionally
interact within the spark chambers according to \eqref{betaMuMinus} and
\eqref{betaMuPlus} to produce muons.
Activity in the spark chamber triggered a camera to produce photographs for
analysis.  To eliminate cosmic ray background, events analyzed were required
to pass a preselection by satisfying two criteria--- that the muon track must
start and stop well within the spark chamber, and that their angle relative to
the beam direction be less than $60^\circ$--- leaving behind 113 events over a
background of 1800.  After surviving preselection, the remaining events were
subdivided into four categories.  Of these events, 49 were labeled as
``short tracks," and eliminated as likely being caused by neutron background.
Another group of 22 were designated ``vertex events"  because they contained
more than one track; not being representative of interactions
\eqref{betaMuMinus} and \eqref{betaMuPlus}, these events were excluded.  Yet
another 9 events were classified as ``showers" due to irregular track activity
uncharacteristic of muons.  Thus, the final selection contained 34 candidate
muon neutrino interactions with a single clean muon track
\cite{numuDiscovery}.
In 1988, Leon Lederman, Melvin Schwartz, Jack Steinberger were awarded the
Nobel Prize in Physics for discovery of the muon neutrino.

The force carriers for the weak force, the \wb and \zb bosons, were
discovered at the Super Proton Synchrotron (SPS) at CERN, thus opening up a new
realm of inquiry for particle physics \cite{wBoson, zBoson}.  Along with the
observation of these bosons came a measurement of their cross sections and
decay rates.  $Z^0$ bosons are expected to decay to neutrinos and contribute to
its decay rate, but these decays are typically unobservable in collider
detectors; this deficit is typically called the invisible rate.  By measuring
the total $Z^0$ decay rate as well as the rate observed through visible
processes, the difference can be taken to obtain the invisible rate.  The
measurement -- made using the ALEPH detector on the Large Electron Positron
collider (LEP) at CERN -- determined of the total number of low-mass
neutrino types that can interact via the weak force.
ALEPH determined that number to
be consistent with three, ruling out the possibility of four
light, active neutrino flavors at the 98\% confidence level \cite{aleph}.
Measurement of
invisible $Z^0$ width, however, only dictates the number of neutrinos that
interact weakly.  Neutrinos that do not interact weakly are known as sterile
neutrinos.

The ALEPH measurement, on top of the previous associations of neutrinos with
electrons and muons, suggested that the third neutrino type should be
associated with taus.
An experiment called Direct Observation of Nu Tau (DONUT) was constructed at
Fermilab to detect $\nu_\tau$ through interactions analogous with the electrons
and muons:
\begin{equation} \label{tauCCMinus}\nu_\tau + n^0 \rightarrow  p^+ + \tau^- \end{equation}
\begin{equation} \label{tauCCPlus}\bar{\nu}_\tau + p^+ \rightarrow n^0 + \tau^+ . \end{equation}
Energetic protons taken from the Tevatron accelerator were directed into a
tungsten beam dump to yield a shower of particles including heavy mesons.  The
primary source of $\nu_\tau$ in the experiment was the $D_S$ meson through its
decay:
\begin{equation}\label{nuTauDS}
D_S \rightarrow \tau + \nu_\tau
\end{equation}
Similar to the Brookhaven muon neutrino experiment, DONUT used an array of
steel absorbers beyond the beam dump, but additionally implemented a set of
deflecting magnets to steer muons and other light particles out of the
beamline.  Downstream of the absorbers was a detector comprised of alternating
layers of two types of tracker modules: emulsion cloud chambers (ECC) and
scintillating fiber trackers (SFT).  The ECC were assembled with sheets of
photographic emulsion that would become exposed along the track of a
fast-moving charged particles.  This detection scheme provided precise spatial
resolution for tracks required for identification of taus by a characteristic
kink in their track 2 mm from the $\nu_\tau$ interaction point.  Interleaved
between the ECC, the SFT was comprised of scintillating fibers that would
produce light in the presence of charged particle radiation and transmit that
light to image intensifiers for digitization.  The SFT was used to identify
regions containing activity within the ECC for analysis.  In 2001 the DONUT
collaboration published the results of this analysis; their paper reported the
observation of four interactions compatible with \eqref{tauCCMinus} and
\eqref{tauCCPlus} above an expected background of 0.34 events  \cite{nuTau}.

\subsection{The Solar Neutrino Problem}

In the 1960's Ray Davis and others set out to determine the flux of neutrinos
incident upon the earth as a result of nuclear fusion within the sun
\cite{davis}.  Davis
and his collaborators designed and built a detector based on a beta process
that converts chlorine atoms to argon atoms, namely:
\begin{equation}\label{nuChlorCap}
{}^{37}Cl + \nu_e  \rightarrow {}^{37}Ar + e^-.
\end{equation}
The cross section for this interaction was calculated in 1964 by John Bahcall
\cite{bahcall}.  Davis' detector was a large tank (placed underground at the
Homestake mine to reduce cosmic ray background) filled with an organic compound
called perchloroethylene\footnote{Perchloroethylene is commonly used as
a dry cleaning agent} ($C_4Cl_8$)  \cite{davis}.
Over time, the solar
neutrino flux converted a measurable fraction of the chlorine atoms to argon.
Every few weeks, Davis would bubble helium through the tanks to collect the
argon atoms and measure the amount collected.  Upon analysis, it was found that
Davis' measured fluxes were a factor of three smaller than those predicted from
solar models.  Many doubted Davis' measurement because it relied on
radiochemical processes rather than event-by-event analysis; however, errors
could not be found in the experiment.  Flux measurements were also made by the
Kamiokande-II experiment and the Sudbury Neutrino Observatory (SNO)
\cite{kamiokande, sno}.
All results were found to be consistent with Davis' measurements but
inconsistent with the prediction.  The deficit in flux measured in these
experiments became known as the ``solar neutrino problem."


The solution to the solar neutrino problem came in the form of neutrino
oscillation, that is, the concept that neutrinos can change flavor as they
propagate.  If neutrinos can change flavor, then some of the electron neutrinos
produced in the sun would transform to muon and tau neutrinos before reaching
earth.  As a result, the aforementioned experiments would indeed measure a
diminished neutrino flux relative to the prediction.

Bruno Pontecorvo first described neutrino oscillation in 1957 as a process
that mixed neutrinos and antineutrinos  \cite{pontecorvo}.  These ideas were
later expanded by Maki, Nakagawa and Sakata in 1962 to explain the small
leptonic decay rate of hyperons\footnote{Hyperon is jargon for a particle
comprised of
one or more strange quarks, but no charm quarks or bottom quarks.} through the
mixing of electron and muon neutrinos \cite{maki1962remarks}.  The theory of
neutrino oscillation was
later expanded to include tau neutrinos.  A detailed description of neutrino
oscillation will come in the following section.


%%%%%%%%%%%%%%%%%%%%%%%%%%%%%%%%%%%%%%%%%%%%%%%%%%%%%%%%%%%%%%%%%%%%%%%%%%%%%%%%
