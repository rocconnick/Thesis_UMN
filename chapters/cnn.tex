%!TEX root = ../thesis_phd.tex
%%%%%%%%%%%%%%%%%%%%%%%%%%%%%%%%%%%%%%%%%%%%%%%%%%%%%%%%%%%%%%%%%%%%%%%%%%%%%%%%
% cnn.tex:
%%%%%%%%%%%%%%%%%%%%%%%%%%%%%%%%%%%%%%%%%%%%%%%%%%%%%%%%%%%%%%%%%%%%%%%%%%%%%%%%
\chapter{Convolutional Neural Network Event Classifier}
\label{cnn_chapter}
%%%%%%%%%%%%%%%%%%%%%%%%%%%%%%%%%%%%%%%%%%%%%%%%%%%%%%%%%%%%%%%%%%%%%%%%%%%%%%%%

At the core of any particle physics analysis is the selection of signal events,
that is, the events which represent the physical process under study.
Traditionally, features are extracted from event reconstruction and passed
to a machine learning alrorithm, e.g. k-nearest neighbor, neural network, or
decision tree.
While this apporach is generally successful, those classifiers can often
be fooled by reconstruction failures.
Even the most robust reconstruction failures will fall victim to
more pathological event topologies.
Some examples include overlapping particle activity which cannot be resolved
and particles which don't travel far enough to make a recongizable track.

An approach which relies on less reconstruction is thus desirable.
In the case of \nova, this means building a classifier which recieves
the raw detector output as input, so as to cut out the reconstruction as a
middleman.

Since \nova detector output is essentially a pair of images with discrete
pixels, it pays to draw inspiration from the computer vision community.
The recent advances in image classification
\cite{krizhevsky2012imagenet,lecun2015deep,szegedy2014going}
discussed in chapter \ref{nnet_chapter} lend themselves well to the task at
hand.

\section{\nova Events as Images}



\section{Architecture}

We use siamese googlenet, two googlenets side-by-side

Train on neutrinos, then add cosmics and fine tune.


\section{Regularization}


\section{Training}




%%%%%%%%%%%%%%%%%%%%%%%%%%%%%%%%%%%%%%%%%%%%%%%%%%%%%%%%%%%%%%%%%%%%%%%%%%%%%%%%
