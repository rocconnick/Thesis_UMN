%!TEX root = ../thesis_phd.tex
%%%%%%%%%%%%%%%%%%%%%%%%%%%%%%%%%%%%%%%%%%%%%%%%%%%%%%%%%%%%%%%%%%%%%%%%%%%%%%%%
% app_glossary.tex: Glossary Appendix:
%%%%%%%%%%%%%%%%%%%%%%%%%%%%%%%%%%%%%%%%%%%%%%%%%%%%%%%%%%%%%%%%%%%%%%%%%%%%%%%%
\chapter{Glossary and Acronyms}
\label{app_glossary}
%%%%%%%%%%%%%%%%%%%%%%%%%%%%%%%%%%%%%%%%%%%%%%%%%%%%%%%%%%%%%%%%%%%%%%%%%%%%%%%%
Care has been taken in this thesis to minimize the use of jargon and
acronyms, but this cannot always be achieved.  This appendix defines
jargon terms in a glossary, and contains a table of acronyms and their
meaning.
%%%%%%%%%%%%%%%%%%%%%%%%%%%%%%%%%%%%%%%%%%%%%%%%%%%%%%%%%%%%%%%%%%%%%%%%%%%%%%%%

%%%%%%%%%%%%%%%%%%%%%%%%%%%%%%%%%%%%%%%%%%%%%%%%%%%%%%%%%%%%%%%%%%%%%%%%%%%%%%%%
% Glossary {{{
%%%%%%%%%%%%%%%%%%%%%%%%%%%%%%%%%%%%%%%%%%%%%%%%%%%%%%%%%%%%%%%%%%%%%%%%%%%%%%%%
\section{Glossary}
\label{jargonapp}
%%%%%%%%%%%%%%%%%%%%%%%%%%%%%%%%%%%%%%%%%%%%%%%%%%%%%%%%%%%%%%%%%%%%%%%%%%%%%%%%
\begin{itemize}

\item \textbf{Analog to Digital Converter} (\textbf{ADC}) -- A device which measures an analog response and produces a discretized value as output.


\item \textbf{Hadron} -- A composite particle made of at least two quarks.  \textit{Mesons} such as pions and kaons are composed of one quark and one antiquark.  \textit{Baryons} are formed by some combination of three quarks or antiquarks.  Recently, more exotic states with four or five quarks have been observed in medium energy electron accelerators \cite{dias2013z_,barth2003evidence}.

\item \textbf{Quantum Efficiency} -- The fraction of photons which produce photo-electrons in a piece of light detection hardware such as a photo-diode or photomultiplier tube.

\item \textbf{Range-out} -- A particle is said to range-out when it runs out of
energy after depositing it all through ionization.

\end{itemize}
%%%%%%%%%%%%%%%%%%%%%%%%%%%%%%%%%%%%%%%%%%%%%%%%%%%%%%%%%%%%%%%%%%%%%%%%%%%%%}}}

%%%%%%%%%%%%%%%%%%%%%%%%%%%%%%%%%%%%%%%%%%%%%%%%%%%%%%%%%%%%%%%%%%%%%%%%%%%%%%%%
% Acronyms {{{
%%%%%%%%%%%%%%%%%%%%%%%%%%%%%%%%%%%%%%%%%%%%%%%%%%%%%%%%%%%%%%%%%%%%%%%%%%%%%%%%
\section{Acronyms}
\label{acronymsec}
%%%%%%%%%%%%%%%%%%%%%%%%%%%%%%%%%%%%%%%%%%%%%%%%%%%%%%%%%%%%%%%%%%%%%%%%%%%%%%%%

%\setlength\LTleft{0pt}
%\setlength\LTright{0pt}

\begin{longtable}{|p{0.25\textwidth}|p{0.75\textwidth}|}
\caption{Acronyms} \label{Acronyms} \\

\hline
Acronym & Meaning \\
\hline \hline
\endfirsthead

\multicolumn{2}{l}%
{{\bfseries \tablename\ \thetable{} -- continued from previous page}} \\
\hline
Acronym & Meaning \\
\hline \hline
\endhead

\hline \hline \multicolumn{2}{|r|}{{Continued on next page}} \\ \hline
\endfoot

\hline \hline
\endlastfoot
CC & Charged current \\
FD & Far Detector \\
LRN & Local response normalization \\
NC & Neutral current \\
ND & Near Detector \\
PVC & polyvinyl chloride \\

\end{longtable}
%%%%%%%%%%%%%%%%%%%%%%%%%%%%%%%%%%%%%%%%%%%%%%%%%%%%%%%%%%%%%%%%%%%%%%%%%%%%%}}}
