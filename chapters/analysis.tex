%!TEX root = ../thesis_phd.tex
%%%%%%%%%%%%%%%%%%%%%%%%%%%%%%%%%%%%%%%%%%%%%%%%%%%%%%%%%%%%%%%%%%%%%%%%%%%%%%%%
% oscillation_analysis.tex: Analysis of Neutrino Oscillations:
%%%%%%%%%%%%%%%%%%%%%%%%%%%%%%%%%%%%%%%%%%%%%%%%%%%%%%%%%%%%%%%%%%%%%%%%%%%%%%%%
\chapter{Analysis}
\label{analysis_chapter}
%%%%%%%%%%%%%%%%%%%%%%%%%%%%%%%%%%%%%%%%%%%%%%%%%%%%%%%%%%%%%%%%%%%%%%%%%%%%%%%%

A measurement of parameters \deltamtht and \thetatth is obtained from
a likelihood fit to the FD data, namely the reconstructed energy spectrum.
Performing such a fit requires a prediction of the FD spectrum which can be
varied in terms of the oscillation parameters.
A naive approach would be to generate a FD prediction straight from
FD Monte Carlo (MC) simulation.
That approach, however, would be quite sensitive to systematic errors,
in particular due to uncertainties  in the \numi flux and neutrino
cross sections.
\nova's two-detector design is aimed precisely at mitigating the effects of
such uncertainties; the ND can be used to determine the expected
spectrum at the FD in a way which mostly cancels the large uncertainties the
the MC prediction.
The procedure of generating the FD prediction in terms of the observed
ND spectrum, referred to as \textit{extrapolation}, is described in Section
\ref{extrap_section}.
The fitting procedure is described in Section \ref{fitting_section}.


\section{Extrapolation}
\label{extrap_section}


Naively, one might expect that the ratio of the FD spectrum to the ND spectrum
could be directly interpreted as an oscillation probability.
This naive approach is complicated by two factors.
First, the ND and FD spectra are independently shaped by various geometric
and other systematic effects.
Second, the imperfect resolution of the energy estimator causes
distortion of the measured spectra which is inconsistent across detectors.
The extrapolation procedure aims to mitigate both of these factors.

Geometric effects form the majority of the discrepancy between the
ND and FD spectra.
Foremost, the ND is a small detector.
The energy spectrum observed by ND is thus shaped by the topologies
of events which can be contained within the detector.
As a result, the ND spectrum is biased towards events with
smaller hadronic showers and shorter muon tracks, although the steel planes
of the muon catcher helps to mitigate the latter problem.
Additionally, even though the ND is considerably smaller than the FD,
it subtends a much larger solid angle relative to the \numi source.
As a result, it sees pions decaying over a wider range of off-axis angle
and thus a broader energy spectrum than the FD.

The geometric effects which shape the ND and FD spectra is well modeled
by the MC simulation.
Selection effects induced by the detector size are handled by simulating
with an accurately sized detector.
The beam acceptance issue is covered well by the beam simulation described in
Section \ref{beam_sim_section}; uncertainties in the beam situation are
treated as a systematic error.
The discrepancy between the FD and ND spectra is shown in Figure
\ref{ratioNDFD}.

\begin{figure}
\begin{center}
  \begin{subfigure}[b]{0.45\textwidth}
    \centering
    \includegraphics[width=\textwidth]{figures/dummy/dummy}
  \end{subfigure}

\end{center}
\caption{FD spectrum, ND spectrum, ratio}{
Need FD and ND spectra, as well as ratio of the two.  Reco?  True?
}
\label{ratioNDFD}
\end{figure}

The extrapolation procedure also aims to capture the smearing effect of
imperfect energy energy resolution independently in each detector.
This is accomplished by forming a 2D histogram of reconstructed energy
vs. true energy in each detector.
These histograms can be used to reweight between the reconstructed
and true spectra in either detector.



\section{Fitting}
\label{fitting_section}

binned maximum likelihood

three flavor model

number of bins

systematic errors and constraints on unmeasured oscillation parameters
are included in the fit with a penalty term

constraints on osc. params taken from PDG

The allowed region is calculated using the FC technique.

Also describe binned maximum likelihood.



%%%%%%%%%%%%%%%%%%%%%%%%%%%%%%%%%%%%%%%%%%%%%%%%%%%%%%%%%%%%%%%%%%%%%%%%%%%%%}}}
