%!TEX root = ../thesis_phd.tex
%%%%%%%%%%%%%%%%%%%%%%%%%%%%%%%%%%%%%%%%%%%%%%%%%%%%%%%%%%%%%%%%%%%%%%%%%%%%%%%%
% abstract.tex: Abstract
%%%%%%%%%%%%%%%%%%%%%%%%%%%%%%%%%%%%%%%%%%%%%%%%%%%%%%%%%%%%%%%%%%%%%%%%%%%%%%%%
\begin{abstract}

The \numi Off-axis Neutrino Experiment (\nova) is designed to study neutrino
oscillations in the \numi beam.
Neutrinos at the Main Injector
(\numi) is currently being upgraded to provide 700 kW.
\nova observes neutrino oscillation using two detectors separated
by a baseline of 810 km;
a 14 kt Far Detector in Ash River, MN and a functionally
identical 0.3 kt Near Detector at Fermilab.
The experiment aims to provide new measurements of \deltamtht and \thetatth
and has potential to determine the neutrino mass hierarchy as well as observe
CP violation in the neutrino sector.
Essential to these analyses is the classification of neutrino
interaction events in \nova detectors.
Raw detector output from \nova is interpretable as a pair
of images which provide orthogonal views of particle interactions.
A recent advance in the field of computer vision
is the advent of convolutional neural networks, which have
delivered top results in the latest image recognition contests.
This work presents a novel approach particle physics analysis
in which a convolutional neural network is used for classification
of particle interactions.
The approach has been demonstrated to improve the signal efficiency and purity
of the event selection, and thus physics sensitivity.
Early \nova data has been analyzed (2.74$\times10^{20}$ POT, 14 kt equivalent)
to provide new best-fit measurements of
$\sin^2(\thetatth) = 0.43$ (with a statistically-degenerate compliment at 0.60)
and
$|\deltamtht| = 2.48\times10^{-3}~\text{eV}^2$.

\end{abstract}

%%%%%%%%%%%%%%%%%%%%%%%%%%%%%%%%%%%%%%%%%%%%%%%%%%%%%%%%%%%%%%%%%%%%%%%%%%%%%%%%
