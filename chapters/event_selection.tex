%!TEX root = ../thesis_phd.tex
%%%%%%%%%%%%%%%%%%%%%%%%%%%%%%%%%%%%%%%%%%%%%%%%%%%%%%%%%%%%%%%%%%%%%%%%%%%%%%%%
% event_selection.tex: Select of showering and tracking events:
%%%%%%%%%%%%%%%%%%%%%%%%%%%%%%%%%%%%%%%%%%%%%%%%%%%%%%%%%%%%%%%%%%%%%%%%%%%%%%%%
\chapter{Event Selection}
\label{event_selection_chapter}
%%%%%%%%%%%%%%%%%%%%%%%%%%%%%%%%%%%%%%%%%%%%%%%%%%%%%%%%%%%%%%%%%%%%%%%%%%%%%%%%

Measuring the parameters which govern a physical process requires
a clean sample of events which exhibit that process.
The definition of clean, of course varies wildly depending on the
experiment and parameters in question.
For this analysis, the goal is to accurately determine the reconstructed energy
spectrum for \numu CC events.
Thus, the goal of event selection is twofold: the selected sample should be
relatively background free and consist of events for which the energy of the
interacting neutrino can be determined.

Estimating the energy of events requires that they be fully contained
within the detector.
If all particles borne in the interaction deposit their energy within
the detector, the a estimate of the neutrino energy can be obtained using
the method described in section \ref{energy_section}.
If one or more of the particles escape the detector, it can be difficult
to estimate how much energy is missing.
Events with poor energy resolution can provide some sensitivity to oscillation
parameters, in particular \thetatth, which governs the amplitude of oscillation.
This analysis, however, makes no attempt at including escaping events.

Two primary backgrounds must be suppressed for this analysis.
The first is the vastly dominant cosmic-ray background; the FD not underground
and is exposed to cosmic rays at a rate in excess of 100 kHz.
The second is source of background is events from the \numi beam which are
not \numu CC interactions.
NC interactions are the trickiest beam background to reject due to their
abundance, especially those with a charged pion in the final state which
can produce a relatively long track.
Both appeared and survived \nue in the beam are relatively rare and tend
not to fake \numu CC signal.

The event classifier described in section \ref{cnn_chapter} serves as to resolve
both cosmic-ray and beam-induced backgrounds.
Additional selection cuts described in this chapter serve to ensure
reconstruction quality and containment.
A handful of extra cuts are required to further reject cosmic background.


\begin{figure}
\begin{center}
\includegraphics[width=0.95\textwidth]{figures/selection/cosmicmuons.png}
\end{center}
\caption{Comparison of track topologies for cosmic-ray and neutrino events}{
Cosmic-ray muon tracks (left) will tend to touch at least one detector face.
Neutrino interactions (right) which occur within the detector will tend to
be well contained.
Cosmic-ray tracks which touch two detector faces can be rejected without
appreciable signal loss.
}

\label{cosmictracks}
\end{figure}



\section{Cosmic Ray Preselection}
\label{cosmicveto_section}

The vast abundance of cosmic ray activity in the FD necessitates the
use of an initial preselection upstream of heavier reconstruction
algorithms.
Running the entire reconstruction chain over all cosmic-ray activity would
simply be too computationally expensive.
Instead, an initial preselection has been applied based on the output
of the slicing and CosmicTrack algorithms described in chapter
\ref{reconstruction_chapter}.
The preselection rejects roughly 90\% of cosmic ray activity while preserving
nearly all contained \numu CC signal.

A large majority of cosmic-ray activity consists of muons which traverse the
detector, as depicted in Figure \ref{cosmictracks}.
If the muon tracks are well reconstructed, this activity can easily be rejected
by identifying tracks which touch two faces of the detector.
Proximity to the detector faces is determined by measuring the distance
between the track start and stop points as projected along the track
direction.
The track projections method is depicted in Figure \ref{trackproj}.
This measure is only reliable when the simple CosmicTrack fit has succeeded,
however.
The fit result is checked by assessing the fraction of hits in the slice
which were included in the track fit, with high values indicating a
successful fit.
Thus, the event is rejected if both projected distances are less than 35 cm and
the hit fraction is greater than 0.8.


\begin{figure}
\begin{center}
\includegraphics[width=0.9\textwidth]{figures/selection/backdist.png}
\end{center}
\caption{Illustration of track distance projection}{
The proximity of a track endpoint to the edge of the detector is determined
by measuring the distance projected along the track direction.
}
\label{trackproj}
\end{figure}

The cosmic-ray preselection also requires events to have at least 20 hits
and fewer than 400.  Events outside of that range are typically of an
interaction energy which carries little oscillation sensitivity.  An additional
selection variable is constructed from the track direction cosines relative to
the beam, $c_{beam}$, and vertical, $c_y$:
\begin{equation}
a = |c_{beam}| * (c_y + 1)
\end{equation}
Tracks from the CosmicTrack algorithm are constructed as downgoing, so $c_y$ is
always negative.  Thus, the variable $a$ is large, the track is generally
directed along the beam axis and not highly vertical.
Events with $a$ greater than 0.2 will pass the preselection.

\section{Reconstruction Quality}

Events of which have not been well reconstructed can cause problems
in energy estimation.
In particular, identification of the muon track within a \numu CC interaction requires the KalmanTrack algorithm to have reconstructed a 3D track.
Events with very little activity are also not especially intelligible.
As such, the reconstruction quality selection requires a reconstructed track
from the CosmicTrack algorithm as well as a 3D track from the KalmanTrack
algorithm.
The quality selection also requires events to have at least 20 hits
spanning at least 4 contiguous planes.

\section{Containment}

\subsection{Far Detector Containment}

Containment in the FD is based on the distance of the slice and track
endpoints from the edges of the detector.
Slices are required to be at within the detector with a buffer of two
planes upstream and downstream as well as two cells in both views.
Track containment is based on projected distances in a similar fashion
to the cosmic-ray preselection, however the distances are measured in
the number of cells spanned by the projection rather than the distance itself.
The track from KalmanTrack is required to have a projected span of at least
11 cells in the forward and backward direction, and at least one cell
for the track from CosmicTrack.

\subsection{Near Detector Containment}

Containment in the ND is more complicated than the FD due to the consideration
of the Muon Catcher and its truncated height.
Slices are rejected if they start in the first two planes of the detector or
end within the last two planes.
Tracks from KalmanTrack is required to start in active region by requiring that
the start $Z$ position is within 1175 cm from the front of the detector.
If a track enters the muon catcher, the height of the track
at the first muon catcher plane is required to be less than the height of
the muon catcher ($Z = 55$ cm) to ensure that it did not leave and re-enter
the detector.
The track projection is also required to span at least five cells in the
forward direction nine cells in the backward direction.
Finally, the hadronic cluster (all hits in the slice not on the muon track)
is required to have calorimetric energy less than 30 MeV.


\section{Background Rejection}

The selections above ensure events are of sufficient quality for analysis
and reduce level of cosmic-ray activity to an acceptable level.
The primary rejection of NC and cosmic-ray events is handled by
the convolutional neural network classifier and some additional requirements
to handle cosmic ray events which would remain in the sample otherwise.

The convolutional neural network discriminant is formed by the sum of the four
softmax outputs corresponding to \numu CC (QE, RES, DIS, other).
Events are rejected if the softmax output is less than 0.5.
Figure \ref{cvnnumu} shows the distribution of the discriminant for
signal and background.
A fair amount of cosmic-ray background slips past the softmax requirement.
The vast majority of these events are near the top of the detector
and can be rejected by requiring the CosmicTrack start $Y$ position to be less
than 600 cm, as seen in Figure \ref{cosStartY}.
Additional cosmic background can be rejected using the transverse momentum
of the event.
Two measures of transverse momentum are used, one which is based
on the reconstructed slices and another based on the reconstructed prongs
described in \cite{niner2015thesis}.
In the slice method, a unit vector is formed in the direction running
from the start of the KalmanTrack to the energy centroid of the slice;
the transverse momentum fraction reconstructed as the sine of the angle
between that vector and the beam axis.
The prong method is similar, except that the the momentum vector
is formed as the energy weighted sum of all reconstructed prongs
for that slice.
Both of these transverse momentum measures, seen in figures \ref{tranMom} and
\ref{pngptp}, are required to be less than 0.8.



\begin{figure}
\begin{center}
\includegraphics[width=0.9\textwidth]{figures/selection/n1_cvnnumu.pdf}
\end{center}
\caption{Distribution of softmax output from convolutional neural network for signal and backgrounds}{
The convolutional neural network output serves as the primary discriminant for
rejecting background.
The blue distribution shows \numu CC signal from the unoscillated beam spectrum,
magenta shows the oscillated spectrum.
The red and green distributions show the cosmic-ray and beam backgrounds,
respectively.
Selection applied to these distributions include all FD analysis cuts
with the exception of the cut on the variable shown here.
Events are rejected if they fall to the left of the vertical
dashed red line.
}
\label{cvnnumu}
\end{figure}


\begin{figure}
\begin{center}
\includegraphics[width=0.9\textwidth]{figures/selection/n1_cosStartY.pdf}
\end{center}
\caption{Distribution of CosmicTrack start $Y$ position for signal and backgrounds}{
The blue distribution shows \numu CC signal from the unoscillated beam spectrum,
magenta shows the oscillated spectrum.
The red and green distributions show the cosmic-ray and beam backgrounds,
respectively.
Selection applied to these distributions include all FD analysis cuts
with the exception of the cut on the variable shown here.
Events are rejected if they fall to the right of the vertical
dashed red line.
}
\label{cosStartY}
\end{figure}

\begin{figure}
\begin{center}
\includegraphics[width=0.9\textwidth]{figures/selection/n1_tranMom.pdf}
\end{center}
\caption{Distribution of transverse momentum estimated from slice mean position
for signal and backgrounds}{
The blue distribution shows \numu CC signal from the unoscillated beam spectrum,
magenta shows the oscillated spectrum.
The red and green distributions show the cosmic-ray and beam backgrounds,
respectively.
Selection applied to these distributions include all FD analysis cuts
with the exception of the cut on the variable shown here.
Events are rejected if they fall to the right of the vertical
dashed red line.
}
\label{pngptp}
\end{figure}

\begin{figure}
\begin{center}
\includegraphics[width=0.9\textwidth]{figures/selection/n1_pngptp.pdf}
\end{center}
\caption{Distribution of transverse momentum estimated from reconstructed prongs
 for signal and backgrounds}{
The blue distribution shows \numu CC signal from the unoscillated beam spectrum,
magenta shows the oscillated spectrum.
The red and green distributions show the cosmic-ray and beam backgrounds,
respectively.
Selection applied to these distributions include all FD analysis cuts
with the exception of the cut on the variable shown here.
Events are rejected if they fall to the right of the vertical
dashed red line.
}
\label{tranMom}
\end{figure}


\begin{figure}
\begin{center}
  \begin{subfigure}[b]{0.7\textwidth}
    \centering
    \includegraphics[width=\textwidth]{figures/selection/myflow_sig_osc.pdf}
  \end{subfigure}

  \begin{subfigure}[b]{0.7\textwidth}
    \centering
    \includegraphics[width=\textwidth]{figures/selection/myflow_sig_eff_osc.pdf}
  \end{subfigure}

\end{center}
\caption{Selected \numu CC signal event spectrum and efficiency}{
The top pane shows the reconstructed neutrino energy spectrum
for selected \numu CC signal events from FD MC simulation.
Truth containment is the sample of events which deposit
no energy outside of the detector, this serves as the denomniator
for the efficiency ratio shown in the bottom pane.
Each subsequent spectrum adds an additional selection cut to the sample.
}
\label{cut_flow_sig}
\end{figure}


\begin{figure}
\begin{center}
  \begin{subfigure}[b]{0.7\textwidth}
    \centering
    \includegraphics[width=\textwidth]{figures/selection/myflow_bkg_osc.pdf}
  \end{subfigure}

  \begin{subfigure}[b]{0.7\textwidth}
    \centering
    \includegraphics[width=\textwidth]{figures/selection/myflow_bkg_eff_osc.pdf}
  \end{subfigure}

\end{center}
\caption{Selected $\nu$ background event spectrum and efficiency}{
The top pane shows the reconstructed neutrino energy spectrum
for selected background events from FD MC simulation.
Truth containment is the sample of events which deposit
no energy outside of the detector, this serves as the denomniator
for the efficiency ratio shown in the bottom pane.
Each subsequent spectrum adds an additional selection cut to the sample.
}
\label{cut_flow_bkg}
\end{figure}

\begin{figure}
\begin{center}
  \begin{subfigure}[b]{0.7\textwidth}
    \centering
    \includegraphics[width=\textwidth]{figures/selection/cosmic_sig_osc.pdf}
  \end{subfigure}

  \begin{subfigure}[b]{0.7\textwidth}
    \centering
    \includegraphics[width=\textwidth]{figures/selection/cosmic_sig_eff_osc.pdf}
  \end{subfigure}

\end{center}
\caption{Comparison of \numu CC signal event spectrum and efficiency to
existing approach}{
The top pane shows the reconstructed neutrino energy spectrum
for selected \numu CC signal events from FD MC simulation.
Truth containment is the sample of events which deposit
no energy outside of the detector, this serves as the denomniator
for the efficiency ratio shown in the bottom pane.
Each subsequent spectrum adds an additional selection cut to the sample.

The two smallest distributions, CNN + Cuts and Muon ID + BDT, compare
the CNN approach to the existing approach used in \cite{nova2016numu}.
The existing approach relied on the Muon ID described in
Section \ref{remid_section}, along with a boosted decision tree
\cite{friedman2002stochastic} to reject cosmic background.
The CNN selection presented here improves the selection efficiency compared to
compared to the existing approach in the low energy bins which are most
sensitive to neutrino oscillation.

}
\label{cut_flow_comp_sig}
\end{figure}
\begin{figure}
\begin{center}
  \begin{subfigure}[b]{0.7\textwidth}
    \centering
    \includegraphics[width=\textwidth]{figures/selection/cosmic_bkg_osc.pdf}
  \end{subfigure}

  \begin{subfigure}[b]{0.7\textwidth}
    \centering
    \includegraphics[width=\textwidth]{figures/selection/cosmic_bkg_eff_osc.pdf}
  \end{subfigure}

\end{center}
\caption{Comparison of $\nu$ background event spectrum and efficiency to
existing approach}{
The top pane shows the reconstructed neutrino energy spectrum
for selected background events from FD MC simulation.
Truth containment is the sample of events which deposit
no energy outside of the detector, this serves as the denomniator
for the efficiency ratio shown in the bottom pane.
Each subsequent spectrum adds an additional selection cut to the sample.

The two smallest distributions, CNN + Cuts and Muon ID + BDT, compare
the CNN approach to the existing approach used in \cite{nova2016numu}.
The existing approach relied on the Muon ID described in
Section \ref{remid_section}, along with a boosted decision tree
\cite{friedman2002stochastic} to reject cosmic background.
The CNN selection presented here improves the background rejection efficiency
compared to the existing approach.
}
\label{cut_flow_comp_sig}
\end{figure}



\begin{figure}
\begin{center}
\includegraphics[width=0.9\textwidth]{figures/selection/contoursfhc1cosmic.pdf}
\end{center}
\caption{Sensitivity comparison between existing analysis approach and CNN
technique}{
The contours shown above are expected 90\% confidence limits for \deltamtht
and \thetatth using \nova MC simulation.
No systematic errors are included.
The red contour uses the selection described in \cite{nova2016numu}, the black
is the CNN approach, and the blue is the result of a perfect selection where
events are selected by MC truth.
}
\label{tranMom}
\end{figure}



%%%%%%%%%%%%%%%%%%%%%%%%%%%%%%%%%%%%%%%%%%%%%%%%%%%%%%%%%%%%%%%%%%%%%%%%%%%%%%%%
